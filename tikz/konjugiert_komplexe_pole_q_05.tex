\begin{tikzpicture}
	[
	x=1cm, y=1cm, scale=0.6, font=\footnotesize, >=latex 
	%Voreinstellung für Pfeilspitzen
	]
	
	%Raster im Hintergrund
	%\draw[step=0.1, gray!50!white, very thin] (0,0) grid (5.5,2.5);
	%\draw[step=1, gray!80!white, thin] (0,0) grid (5.5,2.5);
	
	%Länge x Achse
	\draw [-latex] (0,0) -- ++(5.5,0) node[right] {$\omega$};
	
	%Länge y Achse
	\draw [-latex] (0,0) -- ++(0,2.5) node[above] {$\left\lvert H\left(\omega\right) \right\rvert $};
	
	%Zahlen auf y-Achse 
	\foreach \y in {0,1,2}
	\draw[shift={(0,\y)}] (2pt,0pt) -- (-2pt,0pt);
	
	%Zahlen auf x-Achse
	\foreach \x in {0}
	\draw[shift={(\x,0)},color=black] (0pt,2pt) -- (0pt,-2pt);
	
	%gestrichelte linie
	\draw [dashed, gray, thick] (0,1) -- (4,1);		
	\draw [dashed, red, thick] (4,-0.05) -- (4,1);
	
	%Beschriftungen
	\draw [] (0,2) node[left] {$0 \, \deci \bel$};
	\draw [] (0,1) node[left] {$6 \, \deci \bel$};
	\draw [] (4,0) node[red, below] {$\omega_p$};	
	\draw [] (4,1.8) node[] {negative Überhöhung};	

	%Geschwungene Linie
	\draw[thick, blue] plot[smooth] coordinates {(0,2) (1,1.94) (2,1.75) (3,1.44) (4,1) (4.6,0.57) (5.2,0)};
	\draw [-latex, thick] (4,1.6) -- (4,1.1);

	%Punkte

	\fill [red] (4,1) circle (0.06cm);
	
\end{tikzpicture}
